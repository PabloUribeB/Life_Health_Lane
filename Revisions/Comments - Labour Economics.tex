\documentclass[12pt]{article}
\usepackage{comment}
\usepackage{geometry}
\usepackage{amssymb}
\usepackage{amsmath}
\usepackage{amsthm}
\usepackage{epsfig}
\usepackage{graphicx}
\usepackage{graphics}
%\usepackage{float}
\usepackage[capposition=top]{floatrow}
\usepackage{subcaption}
%\usepackage{subfigure}
\usepackage{subcaption}
\usepackage{setspace}
\usepackage{multirow}
\usepackage{color}
\usepackage{lineno}
\usepackage{fullpage}
\usepackage{multicol}
\usepackage[normalem]{ulem} 
\usepackage{makeidx}
\usepackage{xspace}
\usepackage{wrapfig}
%\usepackage[sort]{natbib}
\usepackage{booktabs}
\usepackage{multirow, array}
\usepackage{amsmath}
\usepackage{lmodern}
\usepackage{mathtools}
%\usepackage{threeparttable}
\usepackage{tabularx}
\usepackage{hyperref}
\usepackage{placeins}
\usepackage{enumerate}
\usepackage{pdflscape}
\usepackage{chngcntr}
\usepackage[title]{appendix}
\usepackage{bbm}
%\usepackage[export]{adjustbox}
%\usepackage[table,xcdraw]{xcolor}
\usepackage{xcolor, soul}
\usepackage{apacite}
\usepackage{natbib}
\usepackage{hyperref}
\usepackage{ragged2e}
\usepackage{threeparttable}
\usepackage{soul}
\title{Letter from the Editor and Reviewers’ comments}

\date{June 6 2024}

\begin{document}
\maketitle


\section{Letter from the Editor}

\subsection{Editor}

Ref.: Ms. No. LABECO-D-24-00167
Life in the Health Lane: The Professional Trajectories of Healthcare Workers in Colombia
Labour Economics \\

Dear Dr. Posso, \\

I have received feedback on your submission from the Co-Guest Editor (see below) and two knowledgeable referees. In light of the referees' comments and my own reading of the paper, I have decided that the paper cannot be considered for publication in Labour Economics. For your guidance, the referee reports are appended below after the Co-Guest Editor's comments. I understand that this outcome is disappointing but I hope it will not discourage you from submitting your work to Labour Economics in the future. Thank you for considering Labour Economics as a possible outlet for your work. \\

Yours sincerely,

Marco Francesconi

Editor-in-Chief

Labour Economics

The Journal of the European Association of Labour Economists \\

\textit{Reviewers' comments are provided below or, otherwise, please log on as an author
to the site at: https://www.editorialmanager.com/labeco/ Your username is: cpossosu@gmail.com If you need to retrieve password details, please go to: click here to reset your password }   \\

\subsection{Co-Guest Editor}

Dear Dr. Posso, \\

Thank you for submitting your research to the Labour Economics Special Issue on the Economics of the Healthcare Workforce. I have sent your paper to two reviewers. R1 recommends a straight reject and raises important concerns about the empirical identification. R2 recommends major revision but is concerned about the clarity of the contribution. Unfortunately based on the reviewers' comments and on my own reading of the paper, I am afraid we need to pass on it. After revisions, your work might be a good fit for \textbf{Health Economics, Social Science and Medicine, Health Services Research.} I am sorry to bear this disappointing news, and I wish you good luck for the publication of your work. \\

Osea Giuntella, 

Co-Guest Editor, Labour Economics \\

\textcolor{red}{Tarea: Pablo y Christian revisar (papers y sitio web) por que deberíamos escoger Health Economics, \st{Daniel y Grey revisar por que deberíamos escoger Journal of Health Economics u otro.}}

\section{Comments from the Reviewers}

\subsection{Reviewer 1}

The current study investigates changes in labor and health outcomes among four groups of healthcare professionals in Colombia after graduation. The authors create a rich longitudinal data set that follows individuals over the years by linking several administrative data sources and use the difference-in-differences-type methodology and, overall, find increases in their labor supply measures and wages. The authors also document increases in the utilization of healthcare services. The paper is well-written, and the authors have been careful with their analysis. In the following, I list my comments and concerns:

\begin{enumerate}
    \item Research design: The authors study the effect of graduation for four groups of health professionals on a series of labor and health outcomes. Given that all individuals in the sample were graduated, there is no control group. The authors explain this issue in the paper and propose the Callaway Sant' Anna approach as a solution utilizing a not-yet-treated group as a control group. However, another way to look at this issue is that the design is not really a difference-in-differences design and is a before-and-after design, that compares the outcomes after graduation to those before graduation. Another piece of evidence to support this claim is that in the regression models, the authors control for individual fixed effects. It may be useful to check how results change if you control for the profession (i.e., physicians, nurses, and dentists) fixed effects compared to individual fixed effects. However, this suggestion may not fully address the issue discussed above. I suggest the authors provide a more thorough discussion to mitigate concerns related this issue. \textcolor{red}{\st{Pablo y Christian. Escribir la ecuación del CS y organizar la sección de diseño.}}
    \item Some of the reported results do not make sense. For example, Figure 6 reports the effect of graduation on the probability of being self-employed. In a simple difference-in-difference setting this means that this outcome is compared between two groups of individuals before and after the intervention. However, it should be safe to assume that self-employment is non-existent before graduation. I recommend dropping this analysis from the paper. \textcolor{red}{\st{Daniel. Eliminar estos resultados.}}
    \item The authors present the trajectories in healthcare utilization for all four professions suggesting that nurses are more likely to go to ER and be admitted to hospitals. Could these healthcare utilizations be linked to their labor market outcomes? In other words, is there any relationship between healthcare professionals' healthcare utilization and their labor market outcomes, likely through missed workdays? It is more plausible to use healthcare utilization as an exogenous source of variation affecting labor market outcomes as opposed to graduation. \textcolor{red}{Todos. Ir pensando.}
    \item Probability of hospitalization: Figure 14 suggests that the probability of hospitalization increases in years following the graduation. Could this be due to getting older? Again, the not-so-ideal study design may play a role here. \textcolor{red}{\st{Pablo y Christian explicación de que todos tienen la misma edad (jóvenes profesionales). Daniel meter edad en estadísticas descriptivas.}}
    \item Are there any structural changes in the findings if the authors split the sample into the before-COVID-19 and after-COVID-19 period? Investigating the trajectory of outcomes separately for these two periods may provide some useful insights about the impact of the COVID-19 pandemic on healthcare professionals' burnout. \textcolor{red}{\st{Pablo y Daniel. Hacer estimaciones sin periodos covid (hasta 2019-2)}}
\end{enumerate}

Minor comments:
\begin{itemize}
    \item The order of graph numbers is not correct. For example, the authors first discuss Graph B.5 on page 18 and then explain Graph B.4. If Graph B.5 should be discussed first it is recommended to include it as Graph B.4. \textcolor{red}{\st{Grey. Ordenar estas figuras.}}
\end{itemize}

\subsection{Reviewer 2}  

This document studies the initial trajectories of healthcare professionals in Colombia's
labor market, graduate school, and health. They exploit the rich a census of Colombian recently graduated bacteriologists, nurses, physicians, and dentists. In particular, it allows the authors to track them before and after graduation from college and other levels. As identification strategy, they consider a staggered event study framework.

\begin{enumerate}
    \item My central comment is the need to present in detail the contribution of this study. So far, it looks that it is to characterize the Colombian labor market for physicians. What could be the benefit for a general reader, from other countries, from reading this document? Or is there some innovation on how to measure labor returns? Thinking that you are measuring income differences but also differential health risks. In such case, an 'aggregate' of the returns should be derived. Are you measuring health outcomes for what? To derive which percentage of the "premium" that health care workers receive is related to higher risk? Perhaps a theoretical framework that ensembles all these elements is useful. \textcolor{red}{Grey. Revisar literatura, lo va conversando con todos.} 
    \item I'm not entirely sure if the DiD presented has any causal effect claim: several times the word "impact" is used, which suggest that the authors might want to sell the results like that. Basically, there is a clear selection process, so this is a comparison between those who decide to be treated against those who decide not to be treated at a given time (check for instance the labor and gender literature, that explore this issue in detail when measuring the 'penalty' of having a child on the labor trajectory of women). Or is there anything that it is not explicitly express in the document?  \textcolor{red}{\st{Pablo y Christian. Revisar el texto para ser muy cuidadosos con las expresiones que den a entender algo de inferencia causal.}} 
    \item For instance, if there is a strongly restricted supply of postgraduate "chairs" per year (say 100 apply but only 5 are accepted), then it could be even considered that starting in each date rather than other is almost random for individuals with similar profiles (similar profession, experience, age, gender, etc). If this is the scenario, a synthetic DiD might be a good option.  \textcolor{red}{Ignorar por ahora.} 
    \item Something that I still do not understand is the "pre-period". Are you considering outcomes while the individual is studying? It should be before they enrol the program, right? Otherwise it is just a comparison of income/labor supply of workers vs non-workers. Yet, Section 5.1.3 considers "months" before graduation". This section should be removed as there is little to learn from it and is misleading. \textcolor{red}{\st{Daniel: Invertir la estructura, primero salud, luego retornos laborales y posgrado. Saquemos la 5.1.3}. Grey: explicar contexto estudiantes y por qué pueden trabajar antes de graduarse (referencias).} 
\end{enumerate}

Other comments: 
\begin{itemize}
    \item Section 2.2 is hard to grasp. Perhaps a diagram showing the alternative paths of specialisation would be useful: how long, which are the requirements for entering the jobe market (and associated costs in terms of time and cash). Section 2.3 has a lot of details that seem unnecessary for understanding the analysis and results. Consider summarising this section.  \textcolor{red}{Grey y Daniel: Diagrama. Discutimos después entre todos y \st{mover sección 2.3 al apéndice. En la 2.2 mencionar que se habla del sistema de salud en Colombia (Christian revisa).}}
    \item Income in Colombian pesos and US dollars in hard to track. Please use a fixed exchange rate for all years (to avoid any exchange rate effect on the analysis) to USD to report results. It is not clear is those numbers are in current or constant prices; hopefully they are in constant prices.  \textcolor{red}{\st{Daniel: PPP 2018 dolares todos los outcomes monetarios. Agregar también en el apéndice de la sección de datos la explicación de este cambio. Hacer una versión de todas las gráficas en términos del efecto relativo (dividir por el promedio del control en -1 y multiplicar por 100).}}
    \item It is not a 100\% clear what is the use of section 3.4. It looks better suited to be part of the results section.
    \item Figure 10: what it means "Mean 0" in the legend labels?
\end{itemize}





%\bibliographystyle{apalike}
%\bibliography{Orion.bib}

\end{document}